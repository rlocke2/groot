%%
%% GMU LaTeX PhD Dissertation Format Template
%%
%% Developed by:
%%      Daniel O. Awduche and Christopher A. St. Jean
%%      Communications and Networking Lab
%%      Dept. of Electrical and Computer Engineering
%%
%% Notes on usage can be found in the accompanying USAGE_NOTES.txt file.
%%
%%**********************************************************************
%% Legal Notice:
%% This code is offered as-is without any warranty either
%% expressed or implied; without even the implied warranty of
%% MERCHANTABILITY or FITNESS FOR A PARTICULAR PURPOSE!
%% User assumes all risk.
%% In no event shall any contributor to this code be liable for any damages
%% or losses, including, but not limited to, incidental, consequential, or
%% any other damages, resulting from the use or misuse of any information
%% contained here.
%%**********************************************************************
%%
%% $Id: GMU_dissertation_template.tex,v 1.7 2007/05/02 02:20:38 Owner Exp $
%%

\documentclass[11 pt]{report}

%%  The file ``gmudissertation.sty''  is the GMU latex style file and
%%   should be placed in the same directory as your LaTeX files
\usepackage{gmudissertation}

%%
%% other packages that need to be loaded
%%
\usepackage{graphicx}                    %   for imported graphics
\usepackage{amsmath}                     %%
\usepackage{amsfonts}                    %%  for AMS mathematics
\usepackage{amssymb}                     %%
\usepackage{amsthm}                      %%
\usepackage[normalem]{ulem}              %   a nice standard underline package
\usepackage[noadjust,verbose,sort]{cite} %   arranges reference citations neatly
\usepackage{setspace}                    %   for line spacing commands

%% The file ``mydissertationabbrev.sty'' is an (optional) personalized file that
%% may contain any and all LaTeX command (re)definitions that will be used
%% throughout the document
%\usepackage{mydissertationabbrev}
\usepackage{enumerate,fullpage,nicefrac,amsmath,amssymb,multicol,amsthm, mathrsfs}
\usepackage{pgf,tikz}
\usetikzlibrary{arrows,shapes}

\newtheorem{theorem}{Theorem}[section]
\newtheorem{corollary}[theorem]{Corollary}
\newtheorem{lemma}[theorem]{Lemma}
\newtheorem{conjecture}[theorem]{Conjecture}
\newtheorem{proposition}[theorem]{Proposition}
\newtheorem{question}[theorem]{Question}
\newtheorem{observation}[theorem]{Observation}
\newtheorem{problem}[theorem]{Problem}

\theoremstyle{definition}
\newtheorem{remark}[theorem]{Remark}

\theoremstyle{definition}
\newtheorem{definition}[theorem]{Definition}

\newcommand{\LG}{\mathcal{L}(G)}

\beforedoc

\begin{document}

%% In this section, all of the user-specific fields to be used in the
%% title pages are set
\title{Analysis in the Discrete Setting of an Infinite Graph and\\
            the Discrete Analogue of the Zygmund Space}
\onelinetitle{Analysis in the Discrete Setting of an Infinite Graph and the Discrete Analogue of the Zygmund Space}
\author{Rachel E. Locke}
\degree{Doctor of Philosophy}
\doctype{Dissertation Proposal}
\dept{George Mason University}
\discipline{Mathematics}

\seconddeg{Master of Science}
\seconddegschool{George Mason University}
\seconddegyear{2012}

\firstdeg{Bachelor of Science}
\firstdegschool{Grove City College}
\firstdegyear{2008}

\degreeyear{2014}

% Note: semester name should be written in its full-form. For example, Fall Semester, not just Fall.
\degreesemester{Fall Semester}

\advisor{Dr. Flavia Colonna}

\firstmember{Dr. David Walnut}

\secondmember{Dr. Walter Morris}

\thirdmember{Dr. Robert Allen}

\depthead{Department Head}

% The current dean is Lloyd J. Griffiths
\deanITE{Dean's Name}

%%
%% Introductory pages
%%

% Note: The signature sheet is set according to the requirements of the Volgenau School of
% Information Technology and Engineering. If your college/school requirement is different,
% please make appropriate changes in the "signaturepage" section of gmudissertation.sty file.
\signaturepage

%\titlepage

% copyright technically optional but should be included in to avoid potential pagination problems
%\copyrightpage

%%
%% Dedication page
%%

%\dedicationpage

%\noindent I dedicate this dissertation to ...
%I dedicate this dissertation to ...
%I dedicate this dissertation to ...
%I dedicate this dissertation to ...
%I dedicate this dissertation to ...
%I dedicate this dissertation to ...
%I dedicate this dissertation to ...

%%
%% Acknowledgements
%%

%\acknowledgementspage

%\noindent I would like to thank the following people who made this possible ...
%I would like to thank the following people who made this possible ...

%%
%% Table of contents, list of tables, and lists of figures
%%

\tableofcontents

%\listoftables

%\listoffigures

%%
%% Abstract
%%
\abstractpage


%The first page of the abstract

%% Be sure to leave a line of whitespace immediately before this line!!!!!
%% (If this comment segment runs together with the preceeding text, you might
%%  see the second page of the abstract numbered "0".)
%%
%% If the abstract is more than one page, then place this line PRECISELY
%% at the page break; otherwise, comment it out.  (See note about this line
%% in the usage notes.)
%%
%\abstractmultiplepage

%The second page of the abstract

%%
%%  the main body of the dissertation
%%
\startofchapters

%% include the chapters one by one (or paste the chapter text in directly if desired)
%\include{chapterOne}

\pagebreak

\chapter{Preliminaries and Background}
 
\indent \indent For a complex Banach space $X$ of functions on a set $\Omega$ and a complex-valued function $\psi$ on $\Omega$, the multiplication operator with symbol $\psi$ is defined as the operator $M_\psi f = \psi f$ for each $f \in X$. The study of multiplication operators attempts to tie the operator theoretic properties of $M_\psi$ to the function theoretic properties of the symbol $\psi$. Properties include boundedness, compactness, and isometries. Also inluded are estimates on the bounds of operator norms and essential norms and identification of the spectrum.

\indent Cohen and Colonna made clear the similarity betwen the classical environment of hte unit disk and the discrete environment of a tree in \cite{CC94}. This prompted the start of interest in studying in the discrete setting. In \cite{Tree}, Colonna and Easley introduce the space $\mathcal{L}$ and characterized operators where $\Omega$ is an infinite tree $T$ and $X$ is the space $\mathcal{L}$ of complex-valued Lipschitz functions on $T$ such that $|f(v) - f(u)| \le C d(v,u)$, for all $u,\,v$ vertices of $T$ and $d(u,v)$ is the distance between $u$ and $v$, and $C$ is a constant \cite{Tree}. This is referred to as the Lipschitz space of the tree, and is a discrete analogue of a Banach space of analytic functions on the open unit disk $\mathbb{D}$, the Bloch space. In a similar analogue to the little Bloch space $\mathcal{B}_0 (\mathbb{D})$ consisting of the Bloch functions $f$ such that $|f^\prime (z)| = o(\frac{1}{1-|z|^2})$ as $|z| \to 1$, Colonna and Easley defined the little Lipschitz space $\mathcal{L}_0$ as the supbspace of $\mathcal{L}$ consisting of all function $f$ on $T$ such that $\lim_{|v| \to \infty} Df(v) = 0$ \cite{Tree}. In \cite{Tree}, Colonna and Easley found growth estimates, characterized bounded and compact multiplication operators, found estimates on operator norm and essential operator norms, studied the spectrum, and found the isometries. Additional work was done on weighted infinite trees in \cite{Weighted} and between spaces on infinite trees in \cite{Between}.

\indent Motivated by this work on infinite trees, we wish to extend these results to a study of infinite graphs. The definition of $\beta _f = \sup_{z \neq w} \frac{|f(z) - f(w)|}{d(z,w)}$ allows for the definition of $Df(v)$ in $\mathcal{L}$ and thus a norm in \cite{Weighted}. The problem we run into for trying to generalize this idea to a general infinite graph $G$ is the idea of $d(z,w)$. Unlike in a tree, we do not have a unique path between any two vertices of the graph. This similarly leads to problems defining our analogy to the little Lipschitz space $L_0$, as we do not have an immediate corollary to $Df = |f(v) - f(v^-)|$ for $v \neq o$ and $Df(o)=0$ \cite{Tree}. We therefore need to define a meaning for $d(v,w)$ and $|v|$. We will use $f^\prime (v)$ rather than the notation of $Df$. We prefer this notation for the analogy to the standard derivative. 

\indent In \cite{Tree}, Colonna and Easley were able to show that $\|Df\|_\infty = \sup_{v\neq w} \frac{|f(v)-f(w)|}{d(v,w)}$, and therefore use the expression $\|f\|_\mathcal{L} = |f(o)|+\|Df\|_\infty$ for the norm. Figure 1.1 shows an example of why the quantity $f^\prime$ is not immediately obvious in the case of a general graph. To find $f^\prime(v)=|f(v)-f(v^-)|$, as it is defined in the case of an infinite tree, we need to choose $v^-$, but it is not clear whether $w$ or $y$ is $v^-$. We are hoping to modify the definition of $f^\prime$ so that this idea can still be applied to general infinite graphs.

\definecolor{qqqqff}{rgb}{0.0,0.0,1.0}
\begin{figure}
\begin{center}
\begin{tikzpicture}[line cap=round,line join=round,>=triangle 45,x=1.0cm,y=1.0cm]
\clip(-3.2600000000000002,1.9399999999999993) rectangle (2.540000000000001,6.04);
\draw (-2.14,3.08)-- (-0.74,3.76);
\draw (-0.74,3.76)-- (-0.58,5.18);
\draw (-0.58,5.18)-- (1.08,4.8);
\draw (1.08,4.8)-- (0.9,3.24);
\draw (0.9,3.24)-- (-0.74,3.76);
\draw (-2.26,3.5399999999999996) node[anchor=north west] {o};
\draw (1.2400000000000009,5.04) node[anchor=north west] {v};
\draw (-0.6799999999999996,5.56) node[anchor=north west] {w};
\draw (1.0400000000000007,3.2399999999999993) node[anchor=north west] {y};
\begin{scriptsize}
\draw [fill=qqqqff] (-2.14,3.08) circle (1.5pt);
\draw [fill=qqqqff] (-0.74,3.76) circle (1.5pt);
\draw [fill=qqqqff] (-0.58,5.18) circle (1.5pt);
\draw [fill=qqqqff] (1.08,4.8) circle (1.5pt);
\draw [fill=qqqqff] (0.9,3.24) circle (1.5pt);
\end{scriptsize}
\end{tikzpicture}
\caption{}
\end{center}
\end{figure}



\indent If a general infinite graph is too difficult, we aim to look at specific types of graphs. Possibilities include allowing only a finite number of cycles in the tree in a variety of configurations. In some cases, this allows for a simple change in definitions to allow all the results from \cite{Tree} to apply, but in other cases it is not trivial. Our first goal here is to provide a workable norm for the space $\mathcal{L}(G)$ where $G$ is our new infinite graph.

\indent Operator theory on the Zygmund space, defined as $\mathcal{Z} = \{f \in H(\mathbb{D}): f^\prime \in \mathcal{B}\}$, where $\mathcal{B}$ is the Bloch space mentioned above, has also been studied. Recently, several authors have studied composition and weighted composition operators acting on the Zygmund space. In \cite{Zyg1}, Colonna and Li characterize bounded and compact operators from $H^\infty$, the space of all bounded analytic functions $f$ on $\mathbb{D}$ with norm $\|f\|_\infty = \sup_{z \in \mathbb{D}} |f(z)|$ to $\mathcal{Z}$ and $\mathcal{Z}_0$, the little Zygmund space. In \cite{Zyg2}, Li and Fu characterize generalized weighted composition operators from the Bloch space into the Zygmund space. Fu and Li also study composition operators acting between Zygmund spaces and $\mathcal{Q}_k$ spaces in \cite{Zyg3}.

\indent In this proposal, based on the anaologue between the Bloch space $\mathcal{B}$ and the Lipschitz space $\mathcal{L}$ in the discrete setting of a tree (or a graph), we wish to study the discrete analogue to the Zygmund space. We then have $\|f\|_{\mathcal{Z}} = |f(0)| + \|f^\prime\|_{\mathcal{B}}$. We also look at if $g \in \mathcal{B}$, then $\displaystyle \|g\|_\mathcal{B} = |g(0)|+\sup_{z \in \mathbb{D}}\, (1 - |z|^2)|g^\prime (z)|$. $\mathcal{Z}$ is a Banach space, and we want $\displaystyle \|f\|_\mathcal{Z} = |f(0)|+|f^\prime(0)| + \beta_f$ where $\beta_f = \sup_{z \in \mathbb{D}} \, (1-|z|^2)|f^{\prime\prime}(z)|<\infty$.  We wish to study anaologous properties in this discrete space to those that have been studied in the continuous setting, such as point-evaluation estimates for functions in $\mathcal{Z}$, bounded multiplication operators, compact multiplication operators, isometries and norm estimates.


\chapter{Proposed Thesis Work}

\indent \indent Motivated by the questions and research above, there are two general directions of proposed research. The following are a series of questions for proposed research in both general directions. The thesis will consist of a subset of the questions listed, depending on the difficulty of the questions. 

\section{Infinite Graphs}

\indent \indent The following questions relate to the direction of research indicated above. Let $G$ be a general infinite connected graph. We are going to consider simple graphs for the reasons that loops will always give $|f(v)-f(v)|=0$ and parallel edges do not impact the value of the function at the vertices. This may change in the case in which the edges have weights, which may be future work. Thus for this proposal, we consider $G$ as an infinite, connected, simple graph.

 \indent We define $\LG$ to be the space of complex-valued functions $f$ defined on the vertices of a graph $G$ such that $|f(v) - f(u)| \le C\, d(u,v)$ for all vertices $v, \, u$ of $G$, where $d(v,u)$ is the distance betwen $v$ and $u$, and $C$ is a constant. We define the distance between two vertices to be $d(v,u)$ as the length of a minimum path from $u$ to $v$. We choose a vertex $o$ to be the root of our general graph, and then define $|v| = d(o,v)$. 


\begin{enumerate}
\item Is $\| f \|_{\mathcal{L}(G)} = |f(o)|+\sup_{v \neq w} \frac{|f(v)-f(w)|}{d(v,w)}$ a norm on $\mathcal{L}(G)$?
\item Is $\mathcal{L}(G)$ complete?
\item Is there a better expression for $\| \cdot \|_{\mathcal{L}(G)}$, such as $|f(o)| + \|f^\prime \|_\infty$ ?
\item How can the little Lipschitz space be defined (if at all) on $\mathcal{L}(G)$?
\item What are the growth conditions?
\item Characterize the bounded and compact multiplication operators on $\mathcal{L}(G)$.
\end{enumerate}

\section{Discrete Zygmund space}

\indent \indent For $f \in \mathcal{L}$, the Liptschitz space, define $f^\prime (v) = f(v) - f(v^-)$ for $v \neq 0$ and $f^\prime (o)=0$. Then, using the notation in \cite{Tree}, $Df(v)=|f^{\prime\prime} (v)|$ so that $\|f\|_\mathcal{L} = |f(o)| + \|f^\prime\|_\infty$, for all $f \in \mathcal{L}$. Define $\mathcal{Z}=\{f:T \to \mathbb{C}|f^\prime \in \mathcal{L}\}$. Then $\|f\|_\mathcal{Z} = |f(o)| + |f^\prime(o)| + \|f^{\prime\prime}\|_\infty = |f(o)| + \|f^\prime\|_\mathcal{L}$.

\indent The following is our main result that we hope to prove and extend. 
\begin{conjecture}\label{Zygmund} $\mathcal{Z}$ is a complex Banach space with norm $\|f\|_\mathcal{Z} = |f(o)|+\|f^\prime\|_\mathcal{L}=|f(o)|+\|f\|_\infty$.\end{conjecture} 

\indent The following questions are related to this direction of research.

\begin{enumerate}
\item Prove Conjecture \ref{Zygmund}.

\item Find the point-evaluation estimate for functions in $\mathcal{Z}$.

\item Characterize $M_\psi : \mathcal{Z} \to \mathcal{Z}$ bounded.

\item Determine norm estimates of $M_\psi$.

\item Characterize $M_\psi : \mathcal{Z} \to \mathcal{Z}$ compact.

\item Determine essential norm estimates.

\item Study the spectrum, the point spectrum, the approximate spectrum. 

\item Find the isometries.

\item Characterize the $M_\psi$ bounded below. 

\end{enumerate}

\section{Radial Trees}

\indent \indent For a third possible area of research, we ask the general question: 
\begin{question} How much can results from homogenous trees be generalized to nonhomogeneous trees? \end{question}

\indent The Bloch space of a homogeneous tree was studied in \cite{HomoGen} by Cohen and Colonna. We wish to extend many of the results from this paper to nonhomogeneous trees. Two categories of nonhomogenous trees could be considered, radial trees and semihomogeneous trees. Radial trees would be the first direction of research in this area. 

\chapter{Preliminary Results}

\indent \indent The following consists of preliminary results on the question of inifinite graphs. 

\indent Define a \textbf{cycle} as a path $[v_1, v_2, \dots v_n, v_1]$ with $v_j \neq v_k$ for all $k \neq j$ and $n$ is the \textbf{length} of the cycle.

\indent Define $\rho (v,w) = $inf\{$l(\pi): \pi$ is a path from $v$ to $w$ and $l(\pi)=$ the number of edges on $\pi$\}. 

\indent In the Banach space $\mathcal{L}$, as defined in [1], we change the definition of $$\| f \| = |f(o)| + \text{sup}_{v \neq w} \frac{|f(v)-f(w)|}{d(v,w)} = |f(o)| + \textit{l}_f$$ Then $\textit{l}_f = \|Df\|_\infty = \text{sup}_{v \neq o} |f(v)-f(v^-)|$. This quantity is well-defined for the following types of infinite trees: infinite trees, as discussed in \cite{Tree}, those with a finite number of disjoint odd length cycles, those with a finite number of odd length cycles disjoint except for possibly pairwise sharing a single vertex. Then all the results from \cite{Tree} should follow with this amendment to the definition of the norm. 

\indent Define $G$ to be an infinite tree with the addition of finite disjoint cycles. (Even cycles now allowed.) We first looked at the case when $G$ is one of the above types with the addition of one even cycle. Then we need to look at extending this to a finite set of disjoint cycles. We still define a vertex $o$ as the \lq\lq origin\rq\rq. 

\indent We are now considering the case for which $G$ has a single even cycle. Let $b$ be the vertex in the even cycle with the minimum distance from $o$. Note that it is possible for $o=b$. Then number the rest of the vertices from $b$ as $\{a_1, a_2, ....a_k, c\}$. $c$ is the vertex opposite $b$, which is possible since the cycle is even. Alternate sides for $a_1$ and $a_2$, etc. Then define $T_1 = $ the \lq\lq tree\rq\rq  formed by removing edge $[b, a_1]$ and $T_2 =$ the "tree" formed by removing edge $[b,a_2]$. Then define $D_f = \text{sup}\{s_1, \dots s_k\}$ where $s_i = \text{sup}_{v \in T_i, v \neq o} |f(v) - f(v^-)|$. We use this definition to hopefully extend to finite cycles. 

\indent We define $\| f\|_G = |f(o)| + \text{sup}_{v \neq w} \frac{|f(v)-f(w)|}{d(v,w)} = |f(o)| + l_f$.  First, we believe that $\|f\|_G$ is a norm:

\indent \indent $\|f\|_G \ge 0$ for all $f$ clearly. $\|f\|_G=0$ if and only if $f(v) = f(w)$ for all $v\neq w$, or when $f$ is constant. Also $|f(o)|$ must be zero as well, meaning $f$ is the constant zero function.  $\|cf\|_G = |cf(o)| + \text{sup}_{v \neq w} \frac{|cf(v)-cf(w)|}{d(v,w)} = |c||f(o)| + |c|\text{sup}_{v\neq w} \frac{|f(v)-f(w)|}{d(v,w)} = |c|\|f\|_G$.  Lastly, $\|f+g\|_G=|(f+g)(o)|+\text{sup}_{v\neq w} \frac{|(f+g)(v) - (f+g)(w)|}{d(v,w)} \le |f(o)| + |g(o)| + l_f + l_g = \|f\|_G + \|g\|_G$, and this is a norm.

\indent We want to show that $l_f = D_f$ so you can consider this as a sort of derivative and the norm would be $\|f\|_G=|f(o)|+D_f$. Now we would like to show that $l_f = D_f$ for the case of one even cycle. 



%\include{chapterTwo}



%% Note: appendix is now put before bibliography.
%% include the following directives if there are any appendices
%\appendix
%\appendixeqnumbering
%\include{Appendix}

%%
%%  bibliography
%%

%% list all of the BibTeX files here for the WinEdt project (if applicable)
%GATHER{bibfile.bib}

%% any bibliography style can be used, but IEEEtran.bst is ideally suited to
%% electrical engineering references

\bibliographystyle{IEEEtran}
\bibliography{IEEEfull,bibfile}

\pagebreak

\begin{thebibliography}{1000}
\bibitem{Tree}F. Colonna and G.R. Easley, \textit{Multiplication operators on the Lipschitz Space of a Tree}, Int. Equ. Oper. Theory \textbf{68} (2010), 391-411.

\bibitem{Weighted}R. Allen, F. Colonna, and G. Easley, \textit{Multiplication Operators on the Weighted Lipschitz Space of a Tree}, J. Operator Theory \textbf{69} (2013), 101-123.

\bibitem{Between}R. Allen, F. Colonna, G. Easley, \textit{Multiplication Operators between Lipschitz-Type Spaces on a Tree}, Int J. of Mathematics and Mathematical Sciences \textbf{2011} (2011), 36 pp.

\bibitem{HomoGen}J.M. Cohen and F. Colonna, \textit{The Bloch Space of a Homogeneous Tree'}, Boletin de la Sociedad Matematica Mexicana \textbf{37} (1992), 63-82.

\bibitem{Zyg1}F. Colonna and S. Li, \textit{Weighted composition operators from $H^\infty$ into the Zygmund spaces}, Complex Anal. Oper. Theory 7 \textbf{(2013)}, no. 5, 1402-1512. 

\bibitem{Zyg2}H. Li and X. Fu, \textit{A new characterization of generalized weighted composition operators from the Bloch space into the Zygmund space}, J. Funct. Spaces Appl. \textbf{2013}, 6 pp.

\bibitem{Zyg3}X. Fu and S. Li, \textit{Composition operators from Zygmund spaces into $Q_k$ spaces}, J. Inequal. Appl. \textbf{2013}:175, 9pp. 

\bibitem{CC94}J.M. Cohen and F. Colonna, \textit{Embeddings of trees in the hyperbolic disk}, Complex Variables \textbf{94}, 1994, 311-335.

\bibitem{ACE14}R.F. Allen, F. Colonna, and G.R. Easley, \textit{Composition operators on the Lipschitz space of a tree}, Mediterr. J. Math. 11 \textbf{(2014)}, no.1, 97-108.

\bibitem{ACE12}R.F. Allen, F. Colonna, and G.R. Easley, \textit{Multiplication operators on the iterated logarithmic Lipschitz spaces of a tree}, Mediterr. J. Math. 9 \textbf{(2012)}, no.4, 575-600.
\end{thebibliography}

%%
%% curriculum vitae
%%
%\cvpage

%\noindent Include your \emph{curriculum vitae} here detailing your background,
%education, and professional experience.
\end{document}
