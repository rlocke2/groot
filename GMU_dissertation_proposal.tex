%%
%% GMU LaTeX PhD Dissertation Format Template
%%
%% Developed by:
%%      Daniel O. Awduche and Christopher A. St. Jean
%%      Communications and Networking Lab
%%      Dept. of Electrical and Computer Engineering
%%
%% Notes on usage can be found in the accompanying USAGE_NOTES.txt file.
%%
%%**********************************************************************
%% Legal Notice:
%% This code is offered as-is without any warranty either
%% expressed or implied; without even the implied warranty of
%% MERCHANTABILITY or FITNESS FOR A PARTICULAR PURPOSE!
%% User assumes all risk.
%% In no event shall any contributor to this code be liable for any damages
%% or losses, including, but not limited to, incidental, consequential, or
%% any other damages, resulting from the use or misuse of any information
%% contained here.
%%**********************************************************************
%%
%% $Id: GMU_dissertation_template.tex,v 1.7 2007/05/02 02:20:38 Owner Exp $
%%

\documentclass[11 pt]{report}

%%  The file ``gmudissertation.sty''  is the GMU latex style file and
%%   should be placed in the same directory as your LaTeX files
\usepackage{gmudissertation}

%%
%% other packages that need to be loaded
%%
\usepackage{graphicx}                    %   for imported graphics
\usepackage{amsmath}                     %%
\usepackage{amsfonts}                    %%  for AMS mathematics
\usepackage{amssymb}                     %%
\usepackage{amsthm}                      %%
\usepackage[normalem]{ulem}              %   a nice standard underline package
\usepackage[noadjust,verbose,sort]{cite} %   arranges reference citations neatly
\usepackage{setspace}                    %   for line spacing commands

%% The file ``mydissertationabbrev.sty'' is an (optional) personalized file that
%% may contain any and all LaTeX command (re)definitions that will be used
%% throughout the document
%\usepackage{mydissertationabbrev}

\beforedoc

\begin{document}

%% In this section, all of the user-specific fields to be used in the
%% title pages are set
\title{First line of the title\\
            second line of the title}
\onelinetitle{The complete title is to be repeated here without any line
        breaks for the second page and for the abstract page}
\author{Rachel E. Locke}
\degree{Doctor of Philosophy}
\doctype{Dissertation Proposal}
\dept{George Mason University}
\discipline{Mathematics}

\seconddeg{Master of Science}
\seconddegschool{My Former School}
\seconddegyear{Year of second degree}

\firstdeg{Bachelor of Science}
\firstdegschool{My Other Former School}
\firstdegyear{Year of first degree}

\degreeyear{Year}

% Note: semester name should be written in its full-form. For example, Fall Semester, not just Fall.
\degreesemester{Semester}

\advisor{Dr. Flavia Colonna}

\firstmember{Dr. David Walnut}

\secondmember{Dr. Walter Morris}

\thirdmember{Third Member}

\depthead{Department Head}

% The current dean is Lloyd J. Griffiths
\deanITE{Dean's Name}

%%
%% Introductory pages
%%

% Note: The signature sheet is set according to the requirements of the Volgenau School of
% Information Technology and Engineering. If your college/school requirement is different,
% please make appropriate changes in the "signaturepage" section of gmudissertation.sty file.
\signaturepage

\titlepage

% copyright technically optional but should be included in to avoid potential pagination problems
\copyrightpage

%%
%% Dedication page
%%

%\dedicationpage

%\noindent I dedicate this dissertation to ...
%I dedicate this dissertation to ...
%I dedicate this dissertation to ...
%I dedicate this dissertation to ...
%I dedicate this dissertation to ...
%I dedicate this dissertation to ...
%I dedicate this dissertation to ...

%%
%% Acknowledgements
%%

%\acknowledgementspage

%\noindent I would like to thank the following people who made this possible ...
%I would like to thank the following people who made this possible ...

%%
%% Table of contents, list of tables, and lists of figures
%%

\tableofcontents

%\listoftables

%\listoffigures

%%
%% Abstract
%%
%\abstractpage

%The first page of the abstract

%% Be sure to leave a line of whitespace immediately before this line!!!!!
%% (If this comment segment runs together with the preceeding text, you might
%%  see the second page of the abstract numbered "0".)
%%
%% If the abstract is more than one page, then place this line PRECISELY
%% at the page break; otherwise, comment it out.  (See note about this line
%% in the usage notes.)
%%
%\abstractmultiplepage

%The second page of the abstract

%%
%%  the main body of the dissertation
%%
\startofchapters

%% include the chapters one by one (or paste the chapter text in directly if desired)
\include{chapterOne}

\textbf{Wish List}\\

\textbf{0.}List the types of graphs for which we change the definition and everything immediately follows\\

\indent \indent Define a \textbf{cycle} as a path $[v_1, v_2, \dots v_n, v_1]$ with $v_j \neq v_k$ for all $k \neq j$ and $n$ is the \textbf{length} of the cycle.\\

\indent \indent Define $\rho (v,w) = $inf\{$l(\pi): \pi$ is a path from $v$ to $w$ and $l(\pi)=$ the number of edges on $\pi$\}. \\

\indent \indent In the Banach space $\mathcal{L}$, as defined in [1], we change the definition of $\| f \| = |f(o)| + \text{sup}_{v \neq w} \frac{|f(v)-f(w)|}{\rho(v,w)} = |f(o)| + \textit{l}_f$. Then $\textit{l}_f = \|Df\|_\infty = \text{sup}_{v \neq o} |f(v)-f(v^-)|$. This quantity is well-defined for the following types of infinite trees: infinite trees, as discussed in [1], those with a finite number of disjoint odd length cycles, those with a finite number of odd length cycles disjoint except for possibly pairwise sharing a single vertex. Then all the results from [1] should follow with this amendment to the definition of the norm. \\

\textbf{1.} Define the new graph $G$ and its spaces better \\

\indent \indent Define $G$ to be an infinite tree with the addition of finite disjoint cycles. (Even cycles now allowed.) We first looked at the case when $G$ is one of the above types with the addition of one even cycle. Then we need to look at extending this to a finite set of disjoint cycles. We still define a vertex $o$ as the "origin". \\

\indent \indent $\mathcal{L}$ is the space of Lipschitz functions for which $\|Df\|_\infty = \text{sup}_{v \in G^*} Df(v) < \infty$ from before, where $G^* = G \backslash o$. I believe that we now need to change this definition. \\

\indent \indent We are now considering the case for which $G$ has a single even cycle. Let $b$ be the vertex in the even cycle with the minimum distance from $o$. Note that it is possible for $o=b$. Then number the rest of the vertices from $b$ as $\{a_1, a_2, ....a_k, c\}$. $c$ is the vertex opposite $b$, which is possible since the cycle is even. Alternate sides for $a_1$ and $a_2$, etc. Then define $T_1 = $ the "tree" formed by removing edge $[b, a_1]$ and $T_2 =$ the "tree" formed by removing edge $[b,a_2]$. Then define $D_f = \text{sup}\{s_1, \dots s_k\}$ where $s_i = \text{sup}_{v \in T_i, v \neq o} |f(v) - f(v^-)|$. We use this definition to hopefully extend to finite cycles. \\

\indent Define the new norm and show it is in fact a norm and the derivative way of writing it works\\

\indent \indent We define $\| f\|_G = |f(o)| + \text{sup}_{v \neq w} \frac{|f(v)-f(w)|}{\rho(v,w)} = |f(o)| + l_f$.  First, $\|f\|_G$ is a norm:\\

\indent \indent \indent $\|f\|_G \ge 0$ for all $f$ clearly. $\|f\|_G=0$ if and only if $f(v) = f(w)$ for all $v\neq w$, or when $f$ is constant. Also $|f(o)|$ must be zero as well, meaning $f$ is the constant zero function.  $\|cf\|_G = |cf(o)| + \text{sup}_{v \neq w} \frac{|cf(v)-cf(w)|}{\rho(v,w)} = |c||f(o)| + |c|\text{sup}_{v\neq w} \frac{|f(v)-f(w)|}{\rho (v,w)} = |c|\|f\|_G$.  Lastly, $\|f+g\|_G=|(f+g)(o)|+\text{sup}_{v\neq w} \frac{|(f+g)(v) - (f+g)(w)|}{\rho(v,w)} \le |f(o)| + |g(o)| + l_f + l_g = \|f\|_G + \|g\|_G$, and this is a norm. \\

\indent \indent We want to show that $l_f = D_f$ so you can consider this as a sort of derivative and the norm would be $\|f\|_G=|f(o)|+D_f$. Now we show that $l_f = D_f$ for the case of one even cycle. 

\indent The space is a functional Banach space\\

\indent Bounded multiplication operators in this space: find a characterization\\

\indent Find upper and lower bounds on the operator norm for a bounded multiplication operator\\

\indent Spectrum?\\

\indent Compact bounded multiplication operators: characterization\\

\indent Essential norm?\\

\indent Isometries: characterize or show nonexistence\\

\textbf{2.} Define a weighted space and norm\\

\indent show it's a Banach space\\

\indent Bounded Multiplication operators characterization\\

\indent upper and lower bound on the norm of bounded multiplication operators\\

\indent Operator Norm bounds\\

\indent Spectrum?\\

\indent Compact operators?\\

\indent Essential norm?\\

\indent Isometries?\\

\textbf{3.} Zygmund space discrete analogue\\

\indent Show $W$ is a Banach space (includes norm definition)\\

\indent Find, if suitable, other formulations for the seminorm\\

\indent Multiplication Operators on $W$\\

\include{chapterTwo}



%% Note: appendix is now put before bibliography.
%% include the following directives if there are any appendices
\appendix
\appendixeqnumbering
\include{Appendix}

%%
%%  bibliography
%%

%% list all of the BibTeX files here for the WinEdt project (if applicable)
%GATHER{bibfile.bib}

%% any bibliography style can be used, but IEEEtran.bst is ideally suited to
%% electrical engineering references

\bibliographystyle{IEEEtran}
\bibliography{IEEEfull,bibfile}

[1] F. Colonna and G.R. Easley, \textit{Multiplication operators on the Lipschitz Space of a Tree}, Int. Equ. Oper. Theory \textbf{68} (2010), 391-411.\\

[2] R. Allen, F. Colonna, and G. Easley, \textit{Multiplication Operators not eh Weighted Lipschitz Space of a Tree}, J. Operator Theory \textbf{69} (2013), 101-123.\\

[3] R. Allen, F. Colonna, G. Easley, \textit{Multiplication Operators between Lipschitz-Type Spaces on a Tree}, Int J. of Mathematics and Mathematical Sciences \textbf{2011} (2011), 36 pp.\\

%%
%% curriculum vitae
%%
%\cvpage

%\noindent Include your \emph{curriculum vitae} here detailing your background,
%education, and professional experience.
\end{document}
